\chapter{Contexto}

\section{Economía Social y Solidaria}
La ESS tiene sus raices en la expansión de iniciativas socioeconómicas autónomas de los sectores populares y sus organizaciones de apoyo como respuesta social a los crecientes niveles de pobreza, exclusión social y precariedad laboral del mundo actual. 

Esas iniciativas sociales han impulsado emprendimientos socioeconómicos como opciones de trabajo, ingresos y búsqueda de mejorar la calidad de vida de sus comunidades de pertenencia, que poseen una matriz identitaria de atributos compartidos, entre los que se destaca el desarrollar actividades económicas con una definida finalidad social (en términos  generales, mejoramiento de las condiciones, ambiente y calidad de vida de sus propios miembros, de algún sector de la sociedad o de la comunidad en un sentido más amplio), a la vez que implican elementos de carácter asociativo y gestión democrática en un contexto de autonomía tanto del sector privado lucrativo como del Estado. 

Algunos ejemplos de estas experiencias son los microemprendimientos, las empresas recuperadas por  sus trabajadores, el cooperativismo de trabajo, las formas de intercambio equitativo (mercado solidario, monedas sociales, etc), los microcréditos y las iniciativas de inserción social como las empresas sociales.





%//////////////////////////////////

\section{¿Qué es el Software Libre?}

El software libre es una cuestión de la libertad  de los  usuarios  de  ejecutar, copiar, distribuir, estudiar, cambiar y mejorar el software. Entre sus beneficios están el ejercicio de la libertad en el uso de la tecnología, el desarrollo de una capacidad de acceso irrestricto y control de las tecnologías (que sólo puede darse usando software libre) y un acceso a la totalidad del código fuente del software utilizado, permitiendo así la posibilidad de observación de la forma de  programar y su funcionamiento.\\

La internacionalización es el proceso mediante el cual se prepara un elemento o producto para permitir su adaptación a diferentes regiones. En el caso de los programas informáticos implica prepararlo para poder ser traducido en varios idiomas, utilizar monedas diferentes o usar distintos formatos de fecha, por citar algunos ejemplos. En algunos casos puede implicar pantallas o procesos de  negocio diferentes. Así se puede decir que un programa informático esta internacionalizado cuando permite su adaptación a diferentes regiones.\\

La internacionalización trae innumerables beneficios como la reducción significativa de la cantidad de entrenamiento necesario para que los usuarios finales puedan utilizar un sistema computacional.\\

Como se dijo, el software libre busca la libertad de acceso y modificación del software. Aunque los autores de un programa hayan previsto numerosas posibilidades de adaptación y adecuación, siempre habrá mas casos en los que será necesario modificarlo, y en el sentido de la internacionalización, los usuarios finales son una parte activa del desarrollo. Esta participación puede darse gracias a que las fuentes del software están disponibles y el conocimiento distribuido: cualquier persona puede traducir una aplicación sin necesidad de obtener un permiso de un propietario, por lo que las lenguas minoritarias se ven muy beneficiadas por este hecho, y también la accesibilidad.\\

Del mismo modo que en el mundo del software libre cuando alguien detecta un problema de seguridad, lo comparte con el resto para así darle solución, la accesibilidad puede verse mejorada de la misma forma. De este modo los usuarios adquieren un rol activo en el desarrollo y mejora de la accesibilidad de aplicación o sistema.